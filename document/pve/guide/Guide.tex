\documentclass[12pt]{beamer}
\usepackage[orientation=portrait,size=custom,width=15.55,height=20,scale=0.5,debug]{beamerposter} 

\usepackage{beamerbasethemes}
\usepackage{calc}
\usepackage{ifthen}
\usepackage[T1]{fontenc}
\usepackage[utf8]{inputenc}
%\usepackage[no-math]{fontspec}
\usepackage{ragged2e}
\usepackage{graphics}
\usepackage[english]{babel}
\usepackage{nicefrac}
\usepackage{xcolor}
\usepackage{calc}
\usepackage{tikz}


\mode<presentation>

\makeatletter
\renewcommand\beamerboxesrounded[2][]{%
    \pgfsetfillopacity{0.5}
    \global\let\beamer@firstlineitemizeunskip=\relax%
    \vbox\bgroup%
    \setkeys{beamerboxes}{upper=block title,lower=block body,width=\textwidth,shadow=false}%
    \setkeys{beamerboxes}{#1}%
    {%
        \usebeamercolor{\bmb@lower}%
        \globalcolorstrue%
        \colorlet{lower.bg}{bg}%
    }%
    {%
        \usebeamercolor{\bmb@upper}%
        \globalcolorstrue%
        \colorlet{upper.bg}{bg}%
    }%
    %
    % Typeset head
    %
    \vskip4bp
    \setbox\bmb@box=\hbox{%
        \begin{minipage}[b]{\bmb@width}%
            \usebeamercolor[fg]{\bmb@upper}%
            #2%
        \end{minipage}}%
        \ifdim\wd\bmb@box=0pt%
        \setbox\bmb@box=\hbox{}%
        \ht\bmb@box=1.5pt%
        \bmb@prevheight=-4.5pt%
        \else%
        \wd\bmb@box=\bmb@width%
        \bmb@temp=\dp\bmb@box%
        \ifdim\bmb@temp<1.5pt%
        \bmb@temp=1.5pt%
        \fi%
        \setbox\bmb@box=\hbox{\raise\bmb@temp\hbox{\box\bmb@box}}%
        \dp\bmb@box=0pt%
        \bmb@prevheight=\ht\bmb@box%
        \fi%
        \bmb@temp=\bmb@width%
        \bmb@dima=\bmb@temp\advance\bmb@dima by2.2bp%
        \bmb@dimb=\bmb@temp\advance\bmb@dimb by4bp%
        \hbox{%
            \begin{pgfpicture}{0bp}{+-\ht\bmb@box}{0bp}{+-\ht\bmb@box}
                \ifdim\wd\bmb@box=0pt%
                \color{lower.bg}%
                \else%
                \color{upper.bg}%
                \fi%
                %               \pgfsetfillopacity{\opacitylevel}%NEW
                \pgfpathqmoveto{-4bp}{-1bp}
                \pgfpathqcurveto{-4bp}{1.2bp}{-2.2bp}{3bp}{0bp}{3bp}
                \pgfpathlineto{\pgfpoint{\bmb@temp}{3bp}}
                \pgfpathcurveto%
                {\pgfpoint{\bmb@dima}{3bp}}%
                {\pgfpoint{\bmb@dimb}{1.2bp}}%
                {\pgfpoint{\bmb@dimb}{-1bp}}%
                \bmb@dima=-\ht\bmb@box%
                \advance\bmb@dima by0.95pt%NEW
                \pgfpathlineto{\pgfpoint{\bmb@dimb}{\bmb@dima}}
                \pgfpathlineto{\pgfpoint{-4bp}{\bmb@dima}}
                \pgfusepath{fill}
            \end{pgfpicture}%
            \copy\bmb@box%
        }%
        \nointerlineskip%
        \vskip-1pt%
        \ifdim\wd\bmb@box=0pt%
        \else%
        \hbox{%
            \begin{pgfpicture}{0pt}{0pt}{\bmb@width}{6pt}
                \bmb@dima=\bmb@width%
                \advance\bmb@dima by8bp%
                \pgfpathrectangle{\pgfpoint{-4bp}{-1bp}}{\pgfpoint{\bmb@dima}{8bp}}
                \pgfusepath{clip}
                {\pgftransformshift{\pgfpoint{-4bp}{0bp}}\pgftext[left,base]{\pgfuseshading{bmb@transition}}}%
            \end{pgfpicture}%
        }%
        \nointerlineskip%
        \vskip-0.5pt%
        \fi%
        \ifbmb@shadow%
        \setbox\bmb@boxshadow=\hbox{\pgfuseshading{bmb@shadow}}%
        \setbox\bmb@boxshadowball=\hbox{\pgfuseshading{bmb@shadowball}}%
        \setbox\bmb@boxshadowballlarge=\hbox{\pgfuseshading{bmb@shadowballlarge}}%
        \fi%
        \setbox\bmb@colorbox=\hbox{{\pgfpicturetrue\pgfsetcolor{lower.bg}}}%
        \setbox\bmb@box=\hbox\bgroup\begin{minipage}[b]{\bmb@width}%
            \vskip2pt%
            \usebeamercolor[fg]{\bmb@lower}%
            \colorlet{beamerstructure}{upper.bg}%
            \colorlet{structure}{upper.bg}%
            %\color{.}%
        }
\makeatother


\setbeamertemplate{blocks}[rounded][shadow=false]
%\addtobeamertemplate{block begin}{\pgfsetfillopacity{0.75}}{\pgfsetfillopacity{1}}


\usepackage{lmodern}
%\usepackage[english]{babel}
%\usepackage[latin1]{inputenc}
%\usepackage[T1]{fontenc}
\usepackage{graphicx} 
\usepackage{textpos} 
\usepackage{multirow}

\setbeamercolor{block title}{fg=structure,bg=white}
%gets rid of bottom navigation bars
\setbeamertemplate{footline}[frame number]{}
%gets rid of bottom navigation symbols
\setbeamertemplate{navigation symbols}{}
%gets rid of footer
\setbeamertemplate{footline}{}
\setbeamertemplate{itemize item}{\raisebox{0.2em}{\scalebox{0.5}{$\blacktriangleright$}}}

\usepackage{pgf,tikz}
\usepackage{mathrsfs}
\usetikzlibrary{arrows}
\pagestyle{empty}
\renewcommand{\baselinestretch}{1.1}

\newcommand*{\colorbar}[2]{
\begin{tikzpicture}[line cap=round,line join=round,>=triangle 45,x=1.0cm,y=1.0cm]\clip(-0.1,-0.1) rectangle (1.8,0.1);
\draw [line width=7.pt,color=#1] (0.,0.)-- (#2/200,0.);
\draw[color=white] (0.2,0.) node {\scriptsize{$#2$}};
\end{tikzpicture}
}
\definecolor{lightgreen}{rgb}{0.,0.6,0.}
\definecolor{lightblue}{rgb}{0.,0.3,1}
\newcommand*{\attack}[1]{\colorbar{red}{#1}}
\newcommand*{\defense}[1]{\colorbar{lightblue}{#1}}
\newcommand*{\stamina}[1]{\colorbar{lightgreen}{#1}}

\newcommand*{\dps}[1]{
\begin{tikzpicture}[line cap=round,line join=round,>=triangle 45,x=1.0cm,y=1.0cm]\clip(-0.1,-0.1) rectangle (1.8,0.1);
\draw [line width=7.pt,color=black] (0.,0.)-- (#1/12.,0.);
\draw[color=white] (0.3,0.) node {\scriptsize{$#1$}};
\end{tikzpicture}
}
\newcommand*{\survival}[1]{
\begin{tikzpicture}[line cap=round,line join=round,>=triangle 45,x=1.0cm,y=1.0cm]\clip(-0.1,-0.1) rectangle (1.8,0.1);
\draw [line width=7.pt,color=black] (0.,0.)-- (#1/25.,0.);
\draw[color=white] (0.3,0.) node {\scriptsize{$#1$}};
\end{tikzpicture}
}
\newcommand*{\tdo}[1]{
\begin{tikzpicture}[line cap=round,line join=round,>=triangle 45,x=1.0cm,y=1.0cm]\clip(-0.1,-0.1) rectangle (1.8,0.1);
\draw [line width=7.pt,color=black] (0.,0.)-- (#1/390.,0.);
\draw[color=white] (0.3,0.) node {\scriptsize{$#1$}};
\end{tikzpicture}
}

\newcommand{\fightingfull}{\includegraphics[height=0.15cm]{../../images/type/full/Fighting.png}}
\newcommand{\bugfull}{\includegraphics[height=0.15cm]{../../images/type/full/Bug.png}}
\newcommand{\darkfull}{\includegraphics[height=0.15cm]{../../images/type/full/Dark.png}}
\newcommand{\electricfull}{\includegraphics[height=0.15cm]{../../images/type/full/Electric.png}}
\newcommand{\fairyfull}{\includegraphics[height=0.15cm]{../../images/type/full/Fairy.png}}
\newcommand{\firefull}{\includegraphics[height=0.15cm]{../../images/type/full/Fire.png}}
\newcommand{\flyingfull}{\includegraphics[height=0.15cm]{../../images/type/full/Flying.png}}
\newcommand{\ghostfull}{\includegraphics[height=0.15cm]{../../images/type/full/Ghost.png}}
\newcommand{\dragonfull}{\includegraphics[height=0.15cm]{../../images/type/full/Dragon.png}}
\newcommand{\grassfull}{\includegraphics[height=0.15cm]{../../images/type/full/Grass.png}}
\newcommand{\groundfull}{\includegraphics[height=0.15cm]{../../images/type/full/Ground.png}}
\newcommand{\icefull}{\includegraphics[height=0.15cm]{../../images/type/full/Ice.png}}
\newcommand{\normalfull}{\includegraphics[height=0.15cm]{../../images/type/full/Normal.png}}
\newcommand{\psychicfull}{\includegraphics[height=0.15cm]{../../images/type/full/Psychic.png}}
\newcommand{\rockfull}{\includegraphics[height=0.15cm]{../../images/type/full/Rock.png}}
\newcommand{\waterfull}{\includegraphics[height=0.15cm]{../../images/type/full/Water.png}}
\newcommand{\poisonfull}{\includegraphics[height=0.15cm]{../../images/type/full/Poison.png}}
\newcommand{\steelfull}{\includegraphics[height=0.15cm]{../../images/type/full/Steel.png}}

\newcommand{\fightingrev}{\includegraphics[height=0.15cm]{../../images/type/simplified_bis/Fighting.png}}
\newcommand{\bugrev}{\includegraphics[height=0.15cm]{../../images/type/simplified_bis/Bug.png}}
\newcommand{\darkrev}{\includegraphics[height=0.15cm]{../../images/type/simplified_bis/Dark.png}}
\newcommand{\electricrev}{\includegraphics[height=0.15cm]{../../images/type/simplified_bis/Electric.png}}
\newcommand{\fairyrev}{\includegraphics[height=0.15cm]{../../images/type/simplified_bis/Fairy.png}}
\newcommand{\firerev}{\includegraphics[height=0.15cm]{../../images/type/simplified_bis/Fire.png}}
\newcommand{\flyingrev}{\includegraphics[height=0.15cm]{../../images/type/simplified_bis/Flying.png}}
\newcommand{\ghostrev}{\includegraphics[height=0.15cm]{../../images/type/simplified_bis/Ghost.png}}
\newcommand{\dragonrev}{\includegraphics[height=0.15cm]{../../images/type/simplified_bis/Dragon.png}}
\newcommand{\grassrev}{\includegraphics[height=0.15cm]{../../images/type/simplified_bis/Grass.png}}
\newcommand{\groundrev}{\includegraphics[height=0.15cm]{../../images/type/simplified_bis/Ground.png}}
\newcommand{\icerev}{\includegraphics[height=0.15cm]{../../images/type/simplified_bis/Ice.png}}
\newcommand{\normalrev}{\includegraphics[height=0.15cm]{../../images/type/simplified_bis/Normal.png}}
\newcommand{\psychicrev}{\includegraphics[height=0.15cm]{../../images/type/simplified_bis/Psychic.png}}
\newcommand{\rockrev}{\includegraphics[height=0.15cm]{../../images/type/simplified_bis/Rock.png}}
\newcommand{\waterrev}{\includegraphics[height=0.15cm]{../../images/type/simplified_bis/Water.png}}
\newcommand{\poisonrev}{\includegraphics[height=0.15cm]{../../images/type/simplified_bis/Poison.png}}
\newcommand{\steelrev}{\includegraphics[height=0.15cm]{../../images/type/simplified_bis/Steel.png}}

\newcommand{\fightingsimp}{\includegraphics[height=0.15cm]{../../images/type/simplified/fighting.png}}
\newcommand{\dragonsimp}{\includegraphics[height=0.15cm]{../../images/type/simplified/dragon.png}}
\newcommand{\darksimp}{\includegraphics[height=0.15cm]{../../images/type/simplified/dark.png}}
\newcommand{\ghostsimp}{\includegraphics[height=0.15cm]{../../images/type/simplified/ghost.png}}
\newcommand{\psysimp}{\includegraphics[height=0.15cm]{../../images/type/simplified/psy.png}}
\newcommand{\icesimp}{\includegraphics[height=0.15cm]{../../images/type/simplified/ice.png}}
\newcommand{\rocksimp}{\includegraphics[height=0.15cm]{../../images/type/simplified/rock.png}}
\newcommand{\groundsimp}{\includegraphics[height=0.15cm]{../../images/type/simplified/ground.png}}
\newcommand{\electricsimp}{\includegraphics[height=0.15cm]{../../images/type/simplified/electric.png}}
\newcommand{\watersimp}{\includegraphics[height=0.15cm]{../../images/type/simplified/water.png}}
\newcommand{\grasssimp}{\includegraphics[height=0.15cm]{../../images/type/simplified/grass.png}}
\newcommand{\firesimp}{\includegraphics[height=0.15cm]{../../images/type/simplified/fire.png}}
\newcommand{\steelsimp}{\includegraphics[height=0.15cm]{../../images/type/simplified/steel.png}}
\newcommand{\fairysimp}{\includegraphics[height=0.15cm]{../../images/type/simplified/fairy.png}}
\newcommand{\flyingsimp}{\includegraphics[height=0.15cm]{../../images/type/simplified/flying.png}}
\newcommand{\poisonsimp}{\includegraphics[height=0.15cm]{../../images/type/simplified/poison.png}}
\newcommand{\bugsimp}{\includegraphics[height=0.15cm]{../../images/type/simplified/bug.png}}


\newcommand{\stardust}{\includegraphics[height=0.15cm]{../../images/objects/stardust}}
\newcommand{\candy}{\includegraphics[height=0.15cm]{../../images/objects/candy}}
\newcommand{\diffhead}{\includegraphics[width=0.2cm]{../../images/Badge_Champion_GOLD_01.png}}

\newcommand{\megaevol}{\includegraphics[width=0.2cm]{../../images/megaevolve}}
\newcommand{\twice}{{\footnotesize{($\times$2)}}}

%\setbeamertemplate{frametitle}[default][center]
\setbeamerfont{frametitle}{size=\normalsize,series=\bfseries}
\setbeamercolor{frametitle}{fg=white}
%\setbeamercolor*{frametitle}{fg=black}

\usepackage{ulem}

\begin{document}

\setbeamertemplate{blocks}[rounded][shadow=false]

\addtobeamertemplate{block begin}{\pgfsetfillopacity{0.5}}{\pgfsetfillopacity{1}}

\newenvironment<>{varblock}[2][\textwidth]{%
  \setlength{\textwidth}{#1}
  \begin{actionenv}#3%
    \def\insertblocktitle{#2}%
    \par%
    \usebeamertemplate{block begin}}
  {\par%
    \usebeamertemplate{block end}%
  \end{actionenv}}

\renewcommand{\baselinestretch}{1.3}

\usebackgroundtemplate{\includegraphics[height=\paperheight]{../../images/type/background/lobby}}
\begin{frame}
\frametitle{1. How do Individual Values (IV) work?}

\begin{block}{}
\begin{footnotesize}
\begin{itemize}
  \item Three statistics define a Pok\'emon: the attack statistic defines the damages your Pokémon will do, the defence statistic defines the speed at which your Pokémon will die, the HP is your Pokémon's Health Points.

  \item We consider the example of Mewtwo. Mewtwo's base statistics are:

\begin{center}
\begin{tabular}{rp{2cm}} 
    & \textbf{Mewtwo} \hfill  \psychicfull  \\ 
%    &  \multicolumn{1}{c}{\includegraphics[width=2cm]{../../images/pokemon/Mewtwo}} \\ \hline
  Base\_ATT &  \attack{300} \\
  Base\_DEF & \defense{182} \\
  Base\_HP & \stamina{214} \\ % \hline
\end{tabular}   
\end{center}

(source: \url{https://pokemon.gameinfo.io/fr/pokemon/150-mewtwo}). 

Assume you are very lucky, and you caught a perfect Mewtwo, with (ATK\_IV, DEF\_IV, HP\_IV) = (15, 15, 15). Your perfect Mewtwo's  stats are:
\begin{center}
\begin{tabular}{rp{2cm}} 
  Base\_ATT + ATK\_IV  &  \attack{315} \\
  Base\_DEF + DEF\_IV & \defense{197} \\
  Base\_HP + HP\_IV & \stamina{229} \\ % \hline
\end{tabular}   
\end{center}

  \item Depending on the level of your Pokémon, you will multiply all these stats by a coefficients given in the following table:

\begin{center}
\begin{tabular}{cccccc} 
  Level 20 &  Level 25 &  Level 30 &  Level 35 &  Level 40 \\ \hline
  0.5974 & 0.667934 & 0.7317 & 0.76156384 & 0.7903 \\
\end{tabular}   
\end{center}

This table tells that: at level 25, your Pokemon is at 84.5\% of its capacity; at level 30, your Pokemon is at 92.6\% of its capacity; at level 35, your Pokemon is at 96.4\% of its capacity. Coming back to Mewtwo, if you level up it to level 30, its statistics are finally:
\begin{center}
\begin{tabular}{rp{2cm}} 
ATT =  (ATT + ATK\_IV) at level 30  &  \attack{230} \\
DEF =  (DEF + DEF\_IV) at level 30 & \defense{144} \\
HP =  (HP + HP\_IV) at level 30 & \stamina{167} \\ % \hline
\end{tabular}   
\end{center}

\item CP formula is:
\begin{center}
CP = $\text{floor}\left(\frac{\text{ATT} \times \sqrt{\text{DEF} \times \text{HP}}}{10} \right) = \text{floor}\left( \frac{230 \times \sqrt{144\times 167}}{10}\right) = 3566$ 
\end{center}
\end{itemize}

\end{footnotesize}
\end{block}
\end{frame}


\begin{frame}
\frametitle{2. How to measure the bulkiness of a Pokémon?}

\begin{block}{}
\begin{footnotesize}
\begin{itemize}
  \item We use the \textbf{effective health} (EH) defined by : EH $\propto$ DEF $\times$ HP. The higher the score is, the bulkier the Pok\'emon is. For instance: 
\begin{center}
\begin{tabular}{rp{2cm}p{2cm}} 
    & \textbf{Tangrowth} \hfill  \grassfull &\textbf{Exeggutor (A)} \hfill  \grassfull~\dragonfull    \\ 
    & \multicolumn{1}{c}{\includegraphics[width=1.5cm]{../../images/pokemon/tangrowth.png}} & \multicolumn{1}{c}{\includegraphics[width=1.5cm]{../../images/pokemon/exeggutor_a.png}} \\
  Base\_ATT &  \attack{207} &\attack{230}  \\
  Base\_DEF & \defense{184} &\defense{153}  \\
  Base\_HP & \stamina{225}  & \stamina{216} \\
\end{tabular}   
\end{center}

Tangrowth's base Effective Health is 184$\times$225 = 41400, whereas Exeggutor (A)'s base Effective Health is 153$\times$216 = 33048.
  \item However, it is also interesting to consider resistances when it comes to compare Pok\'emon. 
  
\begin{center}
\begin{tabular}{cc}
Double vulnerability & $\times 0.39$ \\
Vulnerability & $\times 0.63$ \\
Normal & $\times 1$ \\
Simple Resistance & $\times 1.6$  \\ 
Double Resistance & $\times 2.56$ \\
\end{tabular} 
\end{center}
  
  If we use these two Pokémon versus Kyogre (Waterfall + Hydro Pump), Tangrowth's base Effective Health is 41400$\times$1.6 = 66240, whereas Exeggutor (A)'s base Effective Health is 33048$\times$2.56 = 84602.88. Therefore, Exeggutor (A) will be more resistant versus Kyogre (Hydro Pump) than Tangrowth.
\end{itemize}

\end{footnotesize}
\end{block}
\end{frame}



\begin{frame}
\frametitle{3. How to compute the damage realised by a Pokémon?}

\begin{block}{}
\begin{footnotesize}
The damage a Pok\'emon will do to its opponent is given by:
\begin{center}
Damage formula = $\text{floor}\left(\frac{1}{2}\text{Power} \times \frac{\text{ATT}}{\text{OPP\_DEF}}\times\text{Multiplier}\right) +1$
\end{center}

To illustrate this formula we consider an Exeggutor (A) perfect at level 30 launching Solar Beam to a Kyogre at level 40.
\begin{itemize}
  \item $\text{floor}\left(\right)+1$ guarantees that the minimum damage of any attack is 1.
  \item Power is the power of the considered move, for Solar Beam, it is 180.
  \item ATT is the giver's attack statistic. For instance, for Exeggutor (A) at level 30 with 15 as IV\_ATT, it would be (230+15)$\times$0.7317 = 179.
  \item OPP\_DEF is the opponent's defense statistic. For instance, for Kyogre at level 40 with 15 as IV\_DEF, it would be (228+15)$\times$0.7903 = 192.
  \item Multipliers could be STAB, WAB, FAB and Effectiveness.
  \item STAB is an acronym for Same Type Attack Bonus. If a Pokemon uses a move that matches one of its types, then the attack damage gets a $\times$1.2 multiplier.
  \item WAB is an acronym for Weather Attack Bonus. If a Pokemon uses a move that matches one of the boosted types in the current weather, then the attack gets a $\times$1.2 multiplier, same as that of STAB.
  \item FAB is an acronym for Friendship Attack Bonus. Friend bonuses are 3\%, 5\%, 7\% or 10\% depending on your level of friendship.
  \item Effectiveness refers to the multiplier applied to using a "super effective" or "not very effective" move.  

\begin{center}
\begin{tabular}{cc}
Not very effective (at all!) & $\times 0.39$ \\
Not very effective & $\times 0.63$ \\
Normal & $\times 1$ \\
Super effective & $\times 1.6$  \\ 
(Super!) super effective & $\times 2.56$ \\
\end{tabular} 
\end{center}
\end{itemize}

Under no weather boost, no friendship bonus, the damages Exeggutor will do to Kyogre are:
\begin{center}
Damage formula = $\text{floor}\left(\frac{1}{2}\text{Power} \times \frac{\text{ATT}}{\text{OPP\_DEF}}\times\text{Multiplier}\right) +1 = \text{floor}\left(\frac{1}{2}\text{180} \times \frac{\text{179}}{\text{192}}\times\text{1.2}\times\text{1.6}\right) +1 = 161 + 1 = 162$ HP
\end{center}


\end{footnotesize}
\end{block}
\end{frame}




\begin{frame}
\frametitle{4. Which moves to choose for a Pokemon?}

\begin{block}{}
\begin{footnotesize}
\begin{itemize}
  \item Each Pokemon has a fast move and a charged move. Both together define the damages the Pok\'emon will realise.
  \item Each move is defined by the damage it realises, its duration and the energy it generates (for fast moves) or uses (for charged moves).
  \item Charged move energies are defined by the number of bars:
  \begin{center}
\begin{tabular}{ccc}
%3-bar move&  2-bar move & 1-bar move \\
\begin{tikzpicture}[line cap=round,line join=round,>=triangle 45,x=1.0cm,y=1.0cm]
\clip(-0.1,-0.05) rectangle (0.9,0.05);
\draw [line width=2pt] (0.,0.)-- (0.2,0.);
\draw [line width=2pt] (0.3,0.)-- (0.5,0.);
\draw [line width=2pt] (0.6,0.)-- (0.8,0.);
\end{tikzpicture}
&
\begin{tikzpicture}[line cap=round,line join=round,>=triangle 45,x=1.0cm,y=1.0cm]
\clip(-0.1,-0.05) rectangle (0.9,0.05);
\draw [line width=2pt] (0.,0.)-- (0.35,0.);
\draw [line width=2pt] (0.45,0.)-- (0.8,0.);
\end{tikzpicture}
&
\begin{tikzpicture}[line cap=round,line join=round,>=triangle 45,x=1.0cm,y=1.0cm]
\clip(-0.1,-0.05) rectangle (0.9,0.05);
\draw [line width=2pt] (0.,0.)-- (0.8,0.);
\end{tikzpicture} \\  \hline
energy of 33.33& energy of 50 & energy of 100 \\
\end{tabular}
\end{center}


  \item The goal is then to choose the best moves that realise the \textbf{most damages per second}. We consider the example of Metagross: is Meteor Mash or Flash Cannon more interesting? From \url{https://pokemon.gameinfo.io}, we get the following table:
  \begin{center}
\begin{tabular}{ccc}
& \steelsimp~Meteor Mash & \steelsimp~Flash Cannon \\ \hline
Base damage& 100 &100 \\
Move duration & 2600 ms & 2700 ms \\
Damage window &2300 - 2500 ms & 1600 - 2500 ms \\
Energy& -50 & -100  \\ \hline
DPS	& 38.5 & 37 \\
EPS & -19.2 &-37 \\
\end{tabular}
\end{center}

DPS is the damage per second for this move (= Base damage/Move duration), whereas EPS is the energy used per second (= Base damage/Energy).

Both attacks realise 100 as base damage, and have similar Move duration. However Meteor Mash needs less energy than Flash Cannon; a player will launch two Meteor Mash for one Flash Cannon, and realise twice more damage in the same time. This is reflected in the EPS.

Therefore, Meteor Mash is much more interesting than Flash Cannon.
\item We usually compute the global and neutral DPS for each couple (fast, charge) to see the number of damage per second the Pok\'emon realises by combining its fast and charged moves. Computation is complex, as it needs to consider also the damage realised by the opponent which brings energy to the giver. The DPS must also consider  "Same Type Attack Bonus" (STAB, 25\% damage boost of a move when it is the same type as one of the types of the Pokémon using the move). 

\begin{center}
   \href{https://pokemongo.gamepress.gg/tdo-how-calculate-pokemon-ability-outdated}{\beamergotobutton{How DPS is computed?}} \quad \quad 
   \href{https://pokemongo.gamepress.gg/comprehensive-dps-spreadsheet}{\beamergotobutton{Comprehensive DPS/TDO spreadsheet}}
\end{center}

For Metagross, DPS of (Bullet Punch, Meteor Mash) is much higher than DPS of (Bullet Punch, Flash Cannon):
\begin{center}
\begin{tabular}{ccccccc}
Pokemon & Fast Move & Charged Move & DPS & TDO & DPS$^3 \times$ TDO & CP \\ \hline
Metagross	 & \steelsimp~Bullet Punch	& \steelsimp~Meteor Mash&	17.983	&621.7	&3615.5	&3791\\
%Metagross	& \psysimp~Zen Headbutt	& \steelsimp~Meteor Mash	&17.437	&602.8	&3195.7&	3791\\
Metagross	& \steelsimp~Bullet Punch	& \steelsimp~Flash Cannon&	13.563	&468.9	1&169.7&	3791\\
%Metagross	& Bullet Punch	& Psychic	13.475&	465.8	&1139.6	&3791\\
%Metagross	& Zen Headbutt & 	Flash Cannon&	13.394& 463	&1112.6	&3791 \\
\end{tabular}
\end{center}
\end{itemize}

\end{footnotesize}
\end{block}
\end{frame}




\begin{frame}
\frametitle{5. Which Pokemon to choose for levelling up?}

\begin{block}{}
\begin{footnotesize}

They must be chosen based on their:
\begin{itemize}
  \item Damage Per Second (DPS) defined by their fast and charged move: it represents how much damage a Pok\'emon can make in a second.
  \item Effective Health (EH) is computed via the formula (at level 40):
\begin{center}
EH $=$ (base DEF + DEF\_IV)$\times$ (base HP + HP\_IV) $\times \frac{0.7903^2}{900}$
\end{center}
  \item Total Damage Output (TDO) represents how much damage a Pok\'emon can make before it faints. It is computed via TDO = DPS $\times$ EH
  \item  DPS$^3 \times$ TDO, a mathematical measure without any physical sense, that can help to order Pok\'emon.
\end{itemize} 

\begin{center}
   \href{https://pokemongo.gamepress.gg/comprehensive-dps-spreadsheet}{\beamergotobutton{Comprehensive DPS/TDO spreadsheet}}
\end{center}

For instance, if we compare the two best Ghost-type Pokémon:
\begin{center}
\begin{tabular}{ccccccc}
Pokemon & Fast Move & Charged Move & DPS & TDO & DPS$^3 \times$ TDO & CP \\ \hline
Gengar&	\ghostsimp~Lick	&\ghostsimp~Shadow Ball	&18.109	&350.4	&2080.9	&2878\\ 
Giratina (Origin)&	\ghostsimp~Shadow Claw	&\ghostsimp~Shadow Ball&	15.814	&662.8	&2621.4	&3683\\ 
\end{tabular}
\end{center}

Gengar has a monstrous DPS, but low TDO, whereas Giratina (O) has excellent EH and TDO. The global measure  DPS$^3 \times$ TDO recommends Giratina (O).

Note also that weaknesses are not taken in this spreadsheet, as we compute the neutral DPS. If we choose "Mewtwo" with (Confusion, Psychic) as opponent, the spreadsheet becomes:
\begin{center}
\begin{tabular}{ccccccc}
Pokemon & Fast Move & Charged Move & DPS & TDO & DPS$^3 \times$ TDO & CP \\ \hline
Gengar&	\ghostsimp~Lick	&\ghostsimp~Shadow Ball	&27.547& 194.4 & 4063.3& 2878\\ 
Giratina (Origin)&	\ghostsimp~Shadow Claw	&\ghostsimp~Shadow Ball&	26.415	&641.9	&11829.9	&3683\\ 
\end{tabular}
\end{center}

DPS has increased by $\approx 1.6$ (as Mewtwo is weak to Ghost attacks), but Gengar's TDO has also decreased (due to Mewtwo's statistics and moves, and Gengar's psychic weakness) at the same time, making it not viable.
\end{footnotesize}
\end{block}
\end{frame}




\begin{frame}
\frametitle{6. How difficult are Raid bosses?}

\begin{block}{}
\begin{footnotesize}
Raid boss have the attack and defend statistics of a perfect level 40 Pokémon, but their HP depends on the difficulty of the raid:
\begin{center}
\begin{tabular}{ccc}
Difficulty & HP & Time \\ \hline
\diffhead & 600 HP & 180s \\
\diffhead~\diffhead & 1,800 HP & 180s\\
\diffhead~\diffhead~\diffhead & 3,600 HP & 180s\\
\diffhead~\diffhead~\diffhead~\diffhead & 9,000 HP& 180s \\
\diffhead~\diffhead~\diffhead~\diffhead~\diffhead~(Legendary) & 15,000 HP& 300s \\
\diffhead~\diffhead~\diffhead~\diffhead~\diffhead~\diffhead~(Mewtwo) & 22,500 HP& 300s \\
\end{tabular} 
\end{center}

\parbox{0.4\linewidth}{
For example, Kyogre's base statistics are:

\begin{center}
\begin{tabular}{rp{2cm}} 
    & \textbf{Kyogre} \hfill  \waterfull  \\ 
  Base\_ATT &  \attack{270} \\
  Base\_DEF & \defense{228} \\
  Base\_HP & \stamina{205} \\ % \hline
\end{tabular}   
\end{center}
}
\parbox{0.4\linewidth}{
As a raid boss, they will be:

\begin{center}
\begin{tabular}{rp{2cm}} 
    & \textbf{Kyogre} \hfill  \waterfull  \\ 
  ATT &  \attack{225} \\
  DEF & \defense{192} \\
  HP & 15,000 HP \\ % \hline
\end{tabular}   
\end{center}
}

\begin{itemize}
   \item Damages realised by the raid boss are identical to a Pok\'emon level 40 with ATK\_IV = 15.
  \item For a 5-head boss, you must realise 15000/300 = 50 HP per second; same damage to realise per second for a 4-head boss. The main difficulty of a 5-head boss is that it has more HP, and therefore you will need more time, more Pok\'emon and more revives.
  \item Difficulty of a raid boss will mainly depends on \textbf{the boss defence statistic} (the higher is defense is, the more time you will need to defeat it).
  \item \textbf{You need to play on resistance of your Pok\'emon, if you want to use less potion to defeat it}
  \item \textbf{You need to play on the boss vulnerability, if you want to defeat it faster.}
\end{itemize}

\end{footnotesize}
\end{block}
\end{frame}



\begin{frame}
\frametitle{7. How are IV important? What is the optimal level for a Pok\'emon?}

\begin{block}{}
\begin{footnotesize}

Regarding IVs, they depend on the Pok\'emon base statistics. 
\begin{itemize}
  \item If the Pok\'emon has an important base attack statistic, there will be no difference between ATK\_IV = 14 and ATK\_IV = 15. 
  
  It is recommended to that a high ATK\_IV for Pok\'emon with low DPS.
  \item Regarding to bulkiness (DEF\_IV and HP\_IV), the defence statistic influences how fast your Pok\'emon will die, whereas the HP statistic tells you the total number of HP your Pok\'emon will have. Usually, it is more interesting to have a higher DEF\_IV than HP\_IV. 
  
  It is recommended to that a high DEF\_IV and HP\_IV for Pok\'emon with low Effective Health.
\end{itemize}

The more you want to level up your Pok\'emon, the more expansive it will be. Level 35 is fine (where your Pok\'emon is at 96.4\% of its capacity), but level 30 is good too (92.6\% of its capacity). 

\begin{center}
Level 20 $\xrightarrow[\text{\stardust 31,000}]{\text{\candy~28}}$ Level 25 $\xrightarrow[\text{\stardust 44,000}]{\text{\candy~38}}$ Level 30 $\xrightarrow[\text{\stardust 62,000}]{\text{\candy~64}}$ Level 35 $\xrightarrow[\text{\stardust 88,000}]{\text{\candy~118}}$ Level 40
\end{center}
There is an ideal level depending on each Pok\'emon and its opponent: 
\begin{itemize}
  \item above a certain level, it will not be more resistant (its oponent will be able to inflige a certain number of fast and charged attacks until your Pok\'emon dies). It is called the \textit{bulkpoint}.
  \item above a certain level (usually different from the bulkpoint), your Pok\'emon fast attack will do exactly the same damage (it's due to the floor in the damage formula). It is called the \textit{breakpoint}.
\end{itemize}

\textbf{For levelling up a Pok\'emon after level 35}, if base stat $\approx$ 125, the difference between an IV 15 and an IV 14 is \textbf{one level}. If base stat $\approx$ 265, the difference between an IV 15 and an IV 14 is \textbf{one half-level}.  

%\textbf{7. How are computed based statistics for a Pok\'emon?}

%Read \texttt{https://pokemongohub.net/post/meta/pokemon-go-cp-rework-stat-changes-formula-and-raiding-after-the-rework/}

\end{footnotesize}
\end{block}
\end{frame}


\end{document}