\documentclass[12pt]{beamer}
\usepackage[orientation=portrait,size=custom,width=15.55,height=20,scale=0.5,debug]{beamerposter} 

\usepackage{beamerbasethemes}
\usepackage{calc}
\usepackage{ifthen}
\usepackage[T1]{fontenc}
\usepackage[utf8]{inputenc}
%\usepackage[no-math]{fontspec}
\usepackage{ragged2e}
\usepackage{graphics}
\usepackage[english]{babel}
\usepackage{nicefrac}
\usepackage{xcolor}
\usepackage{calc}
\usepackage{tikz}


\mode<presentation>

\makeatletter
\renewcommand\beamerboxesrounded[2][]{%
    \pgfsetfillopacity{0.5}
    \global\let\beamer@firstlineitemizeunskip=\relax%
    \vbox\bgroup%
    \setkeys{beamerboxes}{upper=block title,lower=block body,width=\textwidth,shadow=false}%
    \setkeys{beamerboxes}{#1}%
    {%
        \usebeamercolor{\bmb@lower}%
        \globalcolorstrue%
        \colorlet{lower.bg}{bg}%
    }%
    {%
        \usebeamercolor{\bmb@upper}%
        \globalcolorstrue%
        \colorlet{upper.bg}{bg}%
    }%
    %
    % Typeset head
    %
    \vskip4bp
    \setbox\bmb@box=\hbox{%
        \begin{minipage}[b]{\bmb@width}%
            \usebeamercolor[fg]{\bmb@upper}%
            #2%
        \end{minipage}}%
        \ifdim\wd\bmb@box=0pt%
        \setbox\bmb@box=\hbox{}%
        \ht\bmb@box=1.5pt%
        \bmb@prevheight=-4.5pt%
        \else%
        \wd\bmb@box=\bmb@width%
        \bmb@temp=\dp\bmb@box%
        \ifdim\bmb@temp<1.5pt%
        \bmb@temp=1.5pt%
        \fi%
        \setbox\bmb@box=\hbox{\raise\bmb@temp\hbox{\box\bmb@box}}%
        \dp\bmb@box=0pt%
        \bmb@prevheight=\ht\bmb@box%
        \fi%
        \bmb@temp=\bmb@width%
        \bmb@dima=\bmb@temp\advance\bmb@dima by2.2bp%
        \bmb@dimb=\bmb@temp\advance\bmb@dimb by4bp%
        \hbox{%
            \begin{pgfpicture}{0bp}{+-\ht\bmb@box}{0bp}{+-\ht\bmb@box}
                \ifdim\wd\bmb@box=0pt%
                \color{lower.bg}%
                \else%
                \color{upper.bg}%
                \fi%
                %               \pgfsetfillopacity{\opacitylevel}%NEW
                \pgfpathqmoveto{-4bp}{-1bp}
                \pgfpathqcurveto{-4bp}{1.2bp}{-2.2bp}{3bp}{0bp}{3bp}
                \pgfpathlineto{\pgfpoint{\bmb@temp}{3bp}}
                \pgfpathcurveto%
                {\pgfpoint{\bmb@dima}{3bp}}%
                {\pgfpoint{\bmb@dimb}{1.2bp}}%
                {\pgfpoint{\bmb@dimb}{-1bp}}%
                \bmb@dima=-\ht\bmb@box%
                \advance\bmb@dima by0.95pt%NEW
                \pgfpathlineto{\pgfpoint{\bmb@dimb}{\bmb@dima}}
                \pgfpathlineto{\pgfpoint{-4bp}{\bmb@dima}}
                \pgfusepath{fill}
            \end{pgfpicture}%
            \copy\bmb@box%
        }%
        \nointerlineskip%
        \vskip-1pt%
        \ifdim\wd\bmb@box=0pt%
        \else%
        \hbox{%
            \begin{pgfpicture}{0pt}{0pt}{\bmb@width}{6pt}
                \bmb@dima=\bmb@width%
                \advance\bmb@dima by8bp%
                \pgfpathrectangle{\pgfpoint{-4bp}{-1bp}}{\pgfpoint{\bmb@dima}{8bp}}
                \pgfusepath{clip}
                {\pgftransformshift{\pgfpoint{-4bp}{0bp}}\pgftext[left,base]{\pgfuseshading{bmb@transition}}}%
            \end{pgfpicture}%
        }%
        \nointerlineskip%
        \vskip-0.5pt%
        \fi%
        \ifbmb@shadow%
        \setbox\bmb@boxshadow=\hbox{\pgfuseshading{bmb@shadow}}%
        \setbox\bmb@boxshadowball=\hbox{\pgfuseshading{bmb@shadowball}}%
        \setbox\bmb@boxshadowballlarge=\hbox{\pgfuseshading{bmb@shadowballlarge}}%
        \fi%
        \setbox\bmb@colorbox=\hbox{{\pgfpicturetrue\pgfsetcolor{lower.bg}}}%
        \setbox\bmb@box=\hbox\bgroup\begin{minipage}[b]{\bmb@width}%
            \vskip2pt%
            \usebeamercolor[fg]{\bmb@lower}%
            \colorlet{beamerstructure}{upper.bg}%
            \colorlet{structure}{upper.bg}%
            %\color{.}%
        }
\makeatother


\setbeamertemplate{blocks}[rounded][shadow=false]
%\addtobeamertemplate{block begin}{\pgfsetfillopacity{0.75}}{\pgfsetfillopacity{1}}


\usepackage{lmodern}
%\usepackage[english]{babel}
%\usepackage[latin1]{inputenc}
%\usepackage[T1]{fontenc}
\usepackage{graphicx} 
\usepackage{textpos} 
\usepackage{multirow}

\setbeamercolor{block title}{fg=structure,bg=white}
%gets rid of bottom navigation bars
\setbeamertemplate{footline}[frame number]{}
%gets rid of bottom navigation symbols
\setbeamertemplate{navigation symbols}{}
%gets rid of footer
\setbeamertemplate{footline}{}
\setbeamertemplate{itemize item}{\raisebox{0.2em}{\scalebox{0.5}{$\blacktriangleright$}}}
\setbeamertemplate{itemize subitem}{\raisebox{0.2em}{\scalebox{0.5}{$\blacktriangleright$}}}

\usepackage{pgf,tikz}
\usepackage{mathrsfs}
\usetikzlibrary{arrows}
\pagestyle{empty}
\renewcommand{\baselinestretch}{1.1}

\newcommand*{\colorbar}[2]{
\begin{tikzpicture}[line cap=round,line join=round,>=triangle 45,x=1.0cm,y=1.0cm]\clip(-0.15,-0.1) rectangle (1.8,0.1);
\draw [line width=7.pt,color=#1] (0.,0.)-- (#2/200,0.);
\draw[color=white] (0.2,0.) node {\scriptsize{$#2$}};
\end{tikzpicture}
}
\definecolor{lightgreen}{rgb}{0.,0.6,0.}
\definecolor{lightblue}{rgb}{0.,0.3,1}
\newcommand*{\attack}[1]{\colorbar{red}{#1}}
\newcommand*{\defense}[1]{\colorbar{lightblue}{#1}}
\newcommand*{\stamina}[1]{\colorbar{lightgreen}{#1}}

\newcommand*{\dps}[1]{
\begin{tikzpicture}[line cap=round,line join=round,>=triangle 45,x=1.0cm,y=1.0cm]\clip(-0.15,-0.1) rectangle (1.8,0.1);
\draw [line width=7.pt,color=black] (0.,0.)-- (#1/12.,0.);
\draw[color=white] (0.3,0.) node {\scriptsize{$#1$}};
\end{tikzpicture}
}
\newcommand*{\survival}[1]{
\begin{tikzpicture}[line cap=round,line join=round,>=triangle 45,x=1.0cm,y=1.0cm]\clip(-0.15,-0.1) rectangle (1.8,0.1);
\draw [line width=7.pt,color=black] (0.,0.)-- (#1/25.,0.);
\draw[color=white] (0.3,0.) node {\scriptsize{$#1$}};
\end{tikzpicture}
}
\newcommand*{\tdo}[1]{
\begin{tikzpicture}[line cap=round,line join=round,>=triangle 45,x=1.0cm,y=1.0cm]\clip(-0.15,-0.1) rectangle (1.8,0.1);
\draw [line width=7.pt,color=black] (0.,0.)-- (#1/390.,0.);
\draw[color=white] (0.3,0.) node {\scriptsize{$#1$}};
\end{tikzpicture}
}

\newcommand{\fightingfull}{\includegraphics[height=0.2cm]{../../../images/type/full/Fighting.png}}
\newcommand{\bugfull}{\includegraphics[height=0.2cm]{../../../images/type/full/Bug.png}}
\newcommand{\darkfull}{\includegraphics[height=0.2cm]{../../../images/type/full/Dark.png}}
\newcommand{\electricfull}{\includegraphics[height=0.2cm]{../../../images/type/full/Electric.png}}
\newcommand{\fairyfull}{\includegraphics[height=0.2cm]{../../../images/type/full/Fairy.png}}
\newcommand{\firefull}{\includegraphics[height=0.2cm]{../../../images/type/full/Fire.png}}
\newcommand{\flyingfull}{\includegraphics[height=0.2cm]{../../../images/type/full/Flying.png}}
\newcommand{\ghostfull}{\includegraphics[height=0.2cm]{../../../images/type/full/Ghost.png}}
\newcommand{\dragonfull}{\includegraphics[height=0.2cm]{../../../images/type/full/Dragon.png}}
\newcommand{\grassfull}{\includegraphics[height=0.2cm]{../../../images/type/full/Grass.png}}
\newcommand{\groundfull}{\includegraphics[height=0.2cm]{../../../images/type/full/Ground.png}}
\newcommand{\icefull}{\includegraphics[height=0.2cm]{../../../images/type/full/Ice.png}}
\newcommand{\normalfull}{\includegraphics[height=0.2cm]{../../../images/type/full/Normal.png}}
\newcommand{\psychicfull}{\includegraphics[height=0.2cm]{../../../images/type/full/Psychic.png}}
\newcommand{\rockfull}{\includegraphics[height=0.2cm]{../../../images/type/full/Rock.png}}
\newcommand{\waterfull}{\includegraphics[height=0.2cm]{../../../images/type/full/Water.png}}
\newcommand{\poisonfull}{\includegraphics[height=0.2cm]{../../../images/type/full/Poison.png}}
\newcommand{\steelfull}{\includegraphics[height=0.2cm]{../../../images/type/full/Steel.png}}

\newcommand{\fightingrev}{\includegraphics[height=0.2cm]{../../images/type/simplified_bis/Fighting.png}}
\newcommand{\bugrev}{\includegraphics[height=0.2cm]{../../images/type/simplified_bis/Bug.png}}
\newcommand{\darkrev}{\includegraphics[height=0.2cm]{../../images/type/simplified_bis/Dark.png}}
\newcommand{\electricrev}{\includegraphics[height=0.2cm]{../../images/type/simplified_bis/Electric.png}}
\newcommand{\fairyrev}{\includegraphics[height=0.2cm]{../../images/type/simplified_bis/Fairy.png}}
\newcommand{\firerev}{\includegraphics[height=0.2cm]{../../images/type/simplified_bis/Fire.png}}
\newcommand{\flyingrev}{\includegraphics[height=0.2cm]{../../images/type/simplified_bis/Flying.png}}
\newcommand{\ghostrev}{\includegraphics[height=0.2cm]{../../images/type/simplified_bis/Ghost.png}}
\newcommand{\dragonrev}{\includegraphics[height=0.2cm]{../../images/type/simplified_bis/Dragon.png}}
\newcommand{\grassrev}{\includegraphics[height=0.2cm]{../../images/type/simplified_bis/Grass.png}}
\newcommand{\groundrev}{\includegraphics[height=0.2cm]{../../images/type/simplified_bis/Ground.png}}
\newcommand{\icerev}{\includegraphics[height=0.2cm]{../../images/type/simplified_bis/Ice.png}}
\newcommand{\normalrev}{\includegraphics[height=0.2cm]{../../images/type/simplified_bis/Normal.png}}
\newcommand{\psychicrev}{\includegraphics[height=0.2cm]{../../images/type/simplified_bis/Psychic.png}}
\newcommand{\rockrev}{\includegraphics[height=0.2cm]{../../images/type/simplified_bis/Rock.png}}
\newcommand{\waterrev}{\includegraphics[height=0.2cm]{../../images/type/simplified_bis/Water.png}}
\newcommand{\poisonrev}{\includegraphics[height=0.2cm]{../../images/type/simplified_bis/Poison.png}}
\newcommand{\steelrev}{\includegraphics[height=0.2cm]{../../images/type/simplified_bis/Steel.png}}

\newcommand{\fightingsimp}{\includegraphics[height=0.2cm]{../../../images/type/simplified/fighting.png}}
\newcommand{\dragonsimp}{\includegraphics[height=0.2cm]{../../../images/type/simplified/dragon.png}}
\newcommand{\darksimp}{\includegraphics[height=0.2cm]{../../../images/type/simplified/dark.png}}
\newcommand{\ghostsimp}{\includegraphics[height=0.2cm]{../../../images/type/simplified/ghost.png}}
\newcommand{\psysimp}{\includegraphics[height=0.2cm]{../../../images/type/simplified/psy.png}}
\newcommand{\icesimp}{\includegraphics[height=0.2cm]{../../../images/type/simplified/ice.png}}
\newcommand{\rocksimp}{\includegraphics[height=0.2cm]{../../../images/type/simplified/rock.png}}
\newcommand{\groundsimp}{\includegraphics[height=0.2cm]{../../../images/type/simplified/ground.png}}
\newcommand{\electricsimp}{\includegraphics[height=0.2cm]{../../../images/type/simplified/electric.png}}
\newcommand{\watersimp}{\includegraphics[height=0.2cm]{../../../images/type/simplified/water.png}}
\newcommand{\grasssimp}{\includegraphics[height=0.2cm]{../../../images/type/simplified/grass.png}}
\newcommand{\firesimp}{\includegraphics[height=0.2cm]{../../../images/type/simplified/fire.png}}
\newcommand{\steelsimp}{\includegraphics[height=0.2cm]{../../../images/type/simplified/steel.png}}
\newcommand{\fairysimp}{\includegraphics[height=0.2cm]{../../../images/type/simplified/fairy.png}}
\newcommand{\flyingsimp}{\includegraphics[height=0.2cm]{../../../images/type/simplified/flying.png}}
\newcommand{\poisonsimp}{\includegraphics[height=0.2cm]{../../../images/type/simplified/poison.png}}
\newcommand{\bugsimp}{\includegraphics[height=0.2cm]{../../../images/type/simplified/bug.png}}


\newcommand{\stardust}{\includegraphics[height=0.2cm]{../../../images/objects/stardust}}
\newcommand{\candy}{\includegraphics[height=0.2cm]{../../../images/objects/rare_candy}}
\newcommand{\diffhead}{\includegraphics[width=0.2cm]{../../../images/Badge_Champion_GOLD_01.png}}

\newcommand{\megaevol}{\includegraphics[width=0.2cm]{../../../images/megaevolve}}
\newcommand{\twice}{{\footnotesize{($\times$2)}}}

%\setbeamertemplate{frametitle}[default][center]
\setbeamerfont{frametitle}{size=\normalsize,series=\bfseries}
\setbeamercolor{frametitle}{fg=white}
%\setbeamercolor*{frametitle}{fg=black}

\usepackage{ulem}

\begin{document}

\setbeamertemplate{blocks}[rounded][shadow=false]

\addtobeamertemplate{block begin}{\pgfsetfillopacity{0.5}}{\pgfsetfillopacity{1}}

\newenvironment<>{varblock}[2][\textwidth]{%
  \setlength{\textwidth}{#1}
  \begin{actionenv}#3%
    \def\insertblocktitle{#2}%
    \par%
    \usebeamertemplate{block begin}}
  {\par%
    \usebeamertemplate{block end}%
  \end{actionenv}}

\renewcommand{\baselinestretch}{1.3}

\usebackgroundtemplate{\includegraphics[height=\paperheight]{../../../images/type/background/lobby}}
\begin{frame}
\frametitle{1. Comment les IV (Individual Values) fonctionnent?}

\begin{block}{}
\begin{footnotesize}
\begin{itemize}
  \item Trois statistiques définissent un Pok\'emon: la statistique d'attaque (ATK) définit les dégâts qu'un Pokémon va réaliser, la statistique de défense (DEF) définit la vitesse à laquelle un Pok\'emon va mourrir, et les points de vie (PV) ou hit points (HP) répresente l'endurance du Pokémon. 
  
  \item Prenons l'exemple de Mewtwo. Ses statistiques de base sont:

\begin{center}
\begin{tabular}{rp{3cm}} 
    & \textbf{Mewtwo} \hfill  \psychicfull  \\ 
%    &  \multicolumn{1}{c}{\includegraphics[width=2cm]{../../images/pokemon/Mewtwo}} \\ \hline
  Base\_ATK &  \attack{300} \\
  Base\_DEF & \defense{182} \\
  Base\_HP & \stamina{214} \\ % \hline
\end{tabular}   
\end{center}

%(source: \url{https://pokemon.gameinfo.io/fr/pokemon/150-mewtwo}). 

Supposons que vous soyez très chanceux et que vous capturez un Mewtwo parfait avec les IV suivant (ATK\_IV, DEF\_IV, HP\_IV) = (15, 15, 15). Les statistiques de base de votre Mewtwo sont:
\begin{center}
\begin{tabular}{rp{3cm}} 
  ATK =  (Base\_ATK + ATK\_IV)   &  \attack{315} \\
  DEF =  (Base\_DEF + DEF\_IV)  & \defense{197} \\
  HP =  (Base\_HP + HP\_IV) & \stamina{229} \\ % \hline
\end{tabular}   
\end{center}

  \item Dépendant du niveau de votre Mewtwo, vous allez multiplier ces statistiques par un coefficient et prendre la partie entière (floor):

\begin{center}
\begin{tabular}{cccccc} 
  Niveau 20 &  Niveau 25 &  Niveau 30 &  Niveau 35 &  Niveau 40 \\ \hline
  0.5974 & 0.667934 & 0.7317 & 0.76156384 & 0.7903 \\
\end{tabular}   
\end{center}

Si vous maxez votre Mewtwo au niveau 40, ses statistiques seront:
\begin{center}
\begin{tabular}{rp{3cm}} 
ATK au niveau 40  &  \attack{248} \\
DEF au niveau 40 & \defense{155} \\
HP au niveau 40 & \stamina{180} \\ % \hline
\end{tabular}   
\end{center}

Votre ami a capturé un Mewtwo dont les IV sont (ATK\_IV, DEF\_IV, HP\_IV) = (14, 15, 14). Si on applique la même démarche que précédemment, au niveau 40, son Mewtwo aura les statistiques suivantes:
\begin{center}
\begin{tabular}{rp{3cm}} 
ATK au niveau 40  &  \attack{248} \\
DEF au niveau 40 & \defense{155} \\
HP au niveau 40 & \stamina{180} \\ % \hline
\end{tabular}   
\end{center}

Surprenament, les deux Mewtwo (14, 15, 14) et (15, 15, 15) au niveau 40 ont exactement les mêmes statistiques de base.

\item La formule de calcule de la puissance de combat (PC) est:
\begin{center}
CP = $\text{floor}\left(\frac{\text{ATK} \times \sqrt{\text{DEF} \times \text{HP}}}{10} \times \text{coeff}\_\text{level}^2 \right)$
\end{center}

\begin{itemize}
\item \footnotesize Pour Mewtwo (15, 15, 15), au niveau 40, son PC est $\text{floor}\left(\frac{315 \times \sqrt{197 \times 229}}{10} \times 0.7903^2 \right) = 4178$
\item \footnotesize Pour Mewtwo (14, 15, 14), au niveau 40, son PC est $\text{floor}\left(\frac{314 \times \sqrt{197 \times 228}}{10} \times 0.7903^2 \right) = 4156$
\end{itemize}
\end{itemize}

\end{footnotesize}
\end{block}
\end{frame}


\begin{frame}
\frametitle{2. Comment mesurer la résistance d'un Pokémon?}

\begin{block}{}
\begin{footnotesize}
\begin{itemize}
  \item On utilise la vie effective (ou \textbf{effective health}, EH) définie par : EH $\propto$ DEF $\times$ HP. Plus ce score est grand, plus le Pok\'emon est résistant. Par exemple:
\begin{center}
\begin{tabular}{rp{3cm}p{3cm}} 
    & \textbf{Bouldeneu} \hfill  \grassfull &\textbf{Noadkoko (A)} \hfill  \grassfull~\dragonfull    \\ 
    & \multicolumn{1}{c}{\includegraphics[width=1.5cm]{../../../images/pokemon/tangrowth.png}} & \multicolumn{1}{c}{\includegraphics[width=1.5cm]{../../../images/pokemon/exeggutor_a.png}} \\
  Base\_ATK &  \attack{207} &\attack{230}  \\
  Base\_DEF & \defense{184} &\defense{153}  \\
  Base\_HP & \stamina{225}  & \stamina{216} \\
\end{tabular}   
\end{center}

La vie effective de Bouldeneu est 184$\times$225 = 41400, tandis que celle de Noadkoko (A)'s est 153$\times$216 = 33048. Donc Bouldeneu est de base plus résistant que Noadkoko (A).
  \item Cependant, il est aussi important de considérer les adversaires qu'ils vont affronter et leur capacité à résister aux attaques adverses et donc de multiplier la vie effective en conséquence par coefficient de résistance:
  
\begin{center}
\begin{tabular}{cc}
Double faiblesse & $\times 0.39$ \\
Faiblesse & $\times 0.63$ \\
Normal & $\times 1$ \\
Simple résistance & $\times 1.6$  \\ 
Double résistance & $\times 2.56$ \\
\end{tabular} 
\end{center}
  
  Comparons ces deux Pok\'emon face à Kyogre (Cascade + Hydrocanon) :
  \begin{itemize}
  \item \footnotesize La vie effective de Bouldeneu est 41400$\times$1.6 = 66240
  \item\footnotesize La vie effective de Noadkoko (A) est 33048$\times$2.56 = 84602.88
\end{itemize}
Par conséquent, Noadkoko (A) est beaucoup plus résistant face à Kyogre (Hydrocanon) que Bouldeneu.
\end{itemize}

\end{footnotesize}
\end{block}
\end{frame}



\begin{frame}
\frametitle{3. Comment calculer les dégâts réalisés par un Pokémon?}

\begin{block}{}
\begin{footnotesize}
Les dégâts qu'un Pokémon inflige a son opposant sont donnés par la formule suivante:
\begin{center}
Dégâts = $\text{floor}\left(\frac{1}{2}\text{Power} \times \frac{\text{ATT}}{\text{OPP\_DEF}}\times\text{Multiplier}\right) +1$
\end{center}

Pour illustrer cette formule, prenons l'exemple de Noadkoko (A) parfait au niveau 30 lançant un Lance-Soleil contre un Kyogre parfait au niveau 40.
\begin{itemize}
  \item $\text{floor}\left(\right)+1$ garantit que les dégâts de n'importe quelle attack est 1.
  \item Power est la puissance de l'attaque, pour Lance-Soleil elle est de 180.
  \item ATT est la statistique d'attaque de celui qui lance l'attaque. Pour Noadkoko (A) au niveau 30 avec 15 en ATK\_IV, elle est de $\text{floor}((230+15)\times0.7317) = 179$.
  \item OPP\_DEF est la statistique de défense de celui qui reçoit l'attaque. Pour Kyogre au niveau 40, avec 15 en DEF\_IV, elle est de $\text{floor}((228+15)\times0.7903) = 192$.
  \item Multipliers sont des multiplieurs comme  STAB, WAB, FAB et Effectiveness.
  \item STAB est l'acronyme pour Same Type Attack Bonus. Si un Pok\'emon utilise une attaque correspondant à l'un de ses types, alors l'attaque reçoit une multiplicateur de $1.2$.
  \item WAB est l'acronyme pour Weather Attack Bonus. Si un Pok\'emon utilise une attaque du même type que la météo, alors l'attaque reçoit une multiplicateur de $1.2$.
  \item FAB est l'acronyme pour Friendship Attack Bonus. Le multiplicateur est de 1.03, 1.05, 1.07 ou 1.1, suivant le niveau d'amitié.
  \item Effectiveness réfère à un multiplicateur lié à l'efficacité de l'attaque (Super efficace, Peu efficace).

\begin{center}
\begin{tabular}{cc}
(Très) peu efficace & $\times 0.39$ \\
Peu efficace & $\times 0.63$ \\
Normal & $\times 1$ \\
Super efficace & $\times 1.6$  \\ 
(Super!) super efficace & $\times 2.56$ \\
\end{tabular} 
\end{center}
\end{itemize}

Sans boost météo, sans bonus d'amitié, les dégâts de Noadkoko sont:
\begin{align*}
\text{Dégâts} %& = \text{floor}\left(\frac{1}{2}\text{Power} \times \frac{\text{ATT}}{\text{OPP\_DEF}}\times\text{Multiplier}\right) +1  \\
& = \text{floor}\left(\frac{1}{2}\text{180} \times \frac{\text{179}}{\text{192}}\times\text{1.2}\times\text{1.6}\right) +1 \\
& = 161 + 1 = 162~ \text{PV}
\end{align*}

Si Noadkoko est au niveau 40, sa statistique d'attaque serait $\text{floor}((230+15)\times0.7903) = 193$ et les dégâts qu'il infligerait seraient:
\[ \text{Dégâts} =  \text{floor}\left(\frac{1}{2}\text{180} \times \frac{\text{193}}{\text{192}}\times\text{1.2}\times\text{1.6}\right) +1 = 173+1 = 174~ \text{PV} \]
\end{footnotesize}
\end{block}
\end{frame}




\begin{frame}
\frametitle{4. Comment choisir les meilleurs attaques?}

\begin{block}{}
\begin{footnotesize}
\begin{itemize}
  \item Chaque Pok\'emon a une attaque rapide et chargée. Ces deux attaques définissent ensemble les dégâts qu'il va réaliser. Chaque attaque est définit par les dégâts qu'il réalise, sa durée et l'énergie qu'il génère (attaque rapide) ou qu'il utilise (attaque chargée). 
  
 \begin{center}
   \href{https://pokemongo.gamepress.gg/pve-fast-moves}{\beamergotobutton{Attaques rapides}}\quad\quad\quad
   \href{https://pokemongo.gamepress.gg/pve-charge-moves}{\beamergotobutton{Attaques chargées}}
   \end{center}
  \item L'énergie nécessaire pour les attaques chargées est définie par le nombre de barres:
  \begin{center}
\begin{tabular}{ccc}
%3-bar move&  2-bar move & 1-bar move \\
\begin{tikzpicture}[line cap=round,line join=round,>=triangle 45,x=1.0cm,y=1.0cm]
\clip(-0.1,-0.05) rectangle (0.9,0.05);
\draw [line width=2pt] (0.,0.)-- (0.2,0.);
\draw [line width=2pt] (0.3,0.)-- (0.5,0.);
\draw [line width=2pt] (0.6,0.)-- (0.8,0.);
\end{tikzpicture}
&
\begin{tikzpicture}[line cap=round,line join=round,>=triangle 45,x=1.0cm,y=1.0cm]
\clip(-0.1,-0.05) rectangle (0.9,0.05);
\draw [line width=2pt] (0.,0.)-- (0.35,0.);
\draw [line width=2pt] (0.45,0.)-- (0.8,0.);
\end{tikzpicture}
&
\begin{tikzpicture}[line cap=round,line join=round,>=triangle 45,x=1.0cm,y=1.0cm]
\clip(-0.1,-0.05) rectangle (0.9,0.05);
\draw [line width=2pt] (0.,0.)-- (0.8,0.);
\end{tikzpicture} \\  \hline
énergie de 33.33& énergie de 50 & énergie de 100 \\
\end{tabular}
\end{center}

  \item Le but est choisir les meilleurs attaques qui réalises le plus de dégâts par secondes (DPS). Considérons l'exemple de Métalosse pour illustration. Est-il plus intéressant qu'il apprenne Poing Météor ou Luminocanon? \`A partir de \url{https://pokemon.gameinfo.io}, on obtient la table suivante:
  \begin{center}
\begin{tabular}{ccc}
& \steelsimp~Poing Météor & \steelsimp~Luminocanon \\ \hline
Dégâts de base & 100 &100 \\
Durée de l'attaque & 2600 ms & 2700 ms \\
Fenêtre d'attaque &2300 - 2500 ms & 1600 - 2500 ms \\
\'Energie & -50 & -100  \\ \hline
DPS	& 38.5 & 37 \\
EPS & -19.2 &-37 \\
\end{tabular}
\end{center}

DPS représente les dégâts par seconde pour cette attaque (= Dégâts de base/Durée de l'attaque, alors que EPS est l'énergie utilisée par seconde (=  Dégâts de base/Energie).

Les deux attaques ont 100 pour dégâts de base et ont la même la même durée. Cependant, Poing Météor nécessite beaucoup moins d'énergie que Luminocanon; un joueur pourra lancer deux Poing Météor pour un seul Luminoncanon (pour la même énergie), et donc réaliser deux fois plus de dégâts. 

Par conséquent, Poing Météor sera plus intéressant que Luminoncanon.
\item Plus généralement, on calcule le DPS neutre gloabl pour chaque couple (rapide, chargée) et on regarde les dégâts par seconde que le Pok\'emon réalise en combinant ces deux attaques. Les calculs sont complexes et considèrent aussi les dégâts que l'opposant puisse réaliser (puisque l'opposant apporte de l'énergie lorsqu'il lance une attaque chargée).

\begin{center}
   \href{https://pokemongo.gamepress.gg/tdo-how-calculate-pokemon-ability-outdated}{\beamergotobutton{Comment le DPS est calculé?}} \quad \quad 
   \href{https://pokemongo.gamepress.gg/comprehensive-dps-spreadsheet}{\beamergotobutton{Table complète DPS/TDO}}
\end{center}

Pour Métalosse, le DPS de (Pisto-Poing, Poing Météor) est bien plus élevé que celui de (Pisto-Poing, Luminocanon).
\begin{center}
\begin{tabular}{ccccccc}
Pok\'emon & Attaque Rapide & Attaque Chargée & DPS & TDO & DPS$^3 \times$ TDO & CP \\ \hline
Métalosse	 & \steelsimp~Pisto-Poing	& \steelsimp~Poing Météor&	17.983	&621.7	&3615.5	&3791\\
%Metagross	& \psysimp~Zen Headbutt	& \steelsimp~Meteor Mash	&17.437	&602.8	&3195.7&	3791\\
Métalosse	& \steelsimp~Pisto-Poing	& \steelsimp~Luminocanon &	13.563	&468.9	1&169.7&	3791\\
%Metagross	& Bullet Punch	& Psychic	13.475&	465.8	&1139.6	&3791\\
%Metagross	& Zen Headbutt & 	Flash Cannon&	13.394& 463	&1112.6	&3791 \\
\end{tabular}
\end{center}
\end{itemize}

\end{footnotesize}
\end{block}
\end{frame}




\begin{frame}
\frametitle{5. Comment comparer deux Pok\'emon?}

\begin{block}{}
\begin{footnotesize}

Pour comparer deux Pok\'emon, il existe différentes mesures:
\begin{itemize}
  \item les Dégâts Par Seconde (DPS), définies par leurs attaques: ils représentent les dégâts que le Pok\'emon peut réaliser en une seconde.
  \item la Vie Effective (ou Effective Health, EH), représentant la résistance du Pok\'emon, et qui vaut au niveau 40:
\begin{center}
EH $=$ (base DEF + DEF\_IV)$\times$ (base HP + HP\_IV) $\times \frac{0.7903^2}{900}$
\end{center}
  \item le Total des dégâts produits (ou Total Damage Output, TDO) qui représente les dégâts que le Pok\'emon peut réaliser avant qu'il meurt. Il est calculé par la formule TDO = DPS $\times$ EH
  \item le DPS$^3 \times$ TDO, une formule mathématique sans aucun sens mathématique qui permet d'ordonner les Pok\'emon en prenant en compte les mesures précédentes.
\end{itemize} 

\begin{center}
   \href{https://pokemongo.gamepress.gg/comprehensive-dps-spreadsheet}{\beamergotobutton{Table complète DPS/TDO}}
\end{center}

Prenons l'exemple des Pok\'emon~\ghostfull. On souhaite les comparer:
\begin{center}
\begin{tabular}{ccccccc}
Pokémon & Attaque Rapide & Attaque Chargée & DPS & TDO & DPS$^3 \times$ TDO & CP \\ \hline
Ectoplasma &	\ghostsimp~Léchouille	&\ghostsimp~Ball Ombre &18.109	&350.4	&2080.9	&2878\\ 
Giratina (Originelle)&	\ghostsimp~Griffe Ombre	&\ghostsimp~Ball Ombre &	15.814	&662.8	&2621.4	&3683\\ 
\end{tabular}
\end{center}

Ectoplasma a un DPS monstrueux, mais un faible TDO. Au contraire, Giratina (O) a une excellente vie et TDO. La mesure globale DPS$^3 \times$ TDO recommende Giratina (O).

On notera que les faiblesses ne sont pas pris en compte dans cette table, puisqu'on calcule le DPS neutre. Si on choisit Mewtwo avec (Choc Mental, Psyko) comme opposant, la table donne:   
\begin{center}
\begin{tabular}{ccccccc}
Pokémon & Attaque Rapide & Attaque Chargée& DPS & TDO & DPS$^3 \times$ TDO & CP \\ \hline
Ectoplasma&	\ghostsimp~Léchouille	&\ghostsimp~Ball Ombre &27.547& 194.4 & 4063.3& 2878\\ 
Giratina (Originelle)&	\ghostsimp~Griffe Ombre	&\ghostsimp~Ball Ombre &	26.415	&641.9	&11829.9	&3683\\ 
\end{tabular}
\end{center}

Le DPS a augmenté par $\approx 1.6$ (car Mewtwo est faible aux attaques~\ghostfull), mais le TDO de Ectoplasma a baissé (ce qui est essentiellement du aux attaques de Mewtwo et la faiblesse \psychicfull~d'Ectoplasma), ce qui ne le rend pas très viable.
\end{footnotesize}
\end{block}
\end{frame}




\begin{frame}
\frametitle{6. Quelle est la difficulté des boss de raid?}

\begin{block}{}
\begin{footnotesize}

Les boss de raid ont la même statistique d'attaque et de défense qu'un Pok\'emon parfait au niveau 40, mais leurs points de vie dépend de la difficulté du raid:
\begin{center}
\begin{tabular}{ccc}
Difficulté & PV & Temps \\ \hline
\diffhead & 600 PV & 180s \\
\diffhead~\diffhead & 1 800 PV & 180s\\
\diffhead~\diffhead~\diffhead & 3 600 PV & 180s\\
\diffhead~\diffhead~\diffhead~\diffhead & 9 000 PV& 180s \\
\diffhead~\diffhead~\diffhead~\diffhead~\diffhead~(Legendary) & 15 000 PV& 300s \\
\diffhead~\diffhead~\diffhead~\diffhead~\diffhead~\diffhead~(Mewtwo) & 22 500 PV& 300s \\
\end{tabular} 
\end{center}

\parbox{0.5\linewidth}{
Par exemple, les statistiques de base de Kyogre sont:

\begin{center}
\begin{tabular}{rp{3cm}} 
    & \textbf{Kyogre} \hfill  \waterfull  \\ 
  Base\_ATK &  \attack{270} \\
  Base\_DEF & \defense{228} \\
  Base\_HP & \stamina{205} \\ % \hline
\end{tabular}   
\end{center}
}
\parbox{0.4\linewidth}{
Comme boss de raid, ils seront:

\begin{center}
\begin{tabular}{rp{3cm}} 
    & \textbf{Kyogre} \hfill  \waterfull  \\ 
  ATT &  \attack{225} \\
  DEF & \defense{192} \\
  HP & 15,000 HP \\ % \hline
\end{tabular}   
\end{center}
}

\begin{itemize}
  \item Les dégâts réalisés par un boss de raid sont identiques à ceux d'un Pok\'emon niveau 40 avec ATK\_IV = 15.
  \item Pour un boss de raid à 5 têtes, il faut lui retirer au moins 15000/300 = 50 PV par seconde; exactement les même PV/seconde à retirer pour un boss de raid à 4 têtes. La difficulté principale des boss de raid à 5 têtes est que comme il a plus de points de vie, il vaut par conséquent plus de temps pour le tuer, et donc plus de Pok\'emon et de potions/rappels.
  \item La difficulté d'un boss de raid (en terme de nombres de joueurs) dépend essentiellement de sa statistique de défense (plus la défense est élevée, plus il faut de temps pour le tuer). Le nombre minimal de joueurs (avec les meilleurs contres au niveau 30, niveau d'amitié "meilleurs amis" et météo extrême) est:
\begin{itemize}
  \item \footnotesize Si le boss de raid n'a que de simples faiblesses, il est de floor(Base DEF/80)+1.
  \item \footnotesize Si le boss de raid a une double faiblesse, il est de floor(Base DEF/80)+1.
\end{itemize}
  
  \item \textbf{Vous devez jouer sur la résistance de vos Pok\'emon pour utiliser le moins de potions/rappels pour le tuer.}
  \item \textbf{Vous devez jouer sur les faiblesses du boss, si vous voulez le tuer rapidement.}
\end{itemize}

\end{footnotesize}
\end{block}
\end{frame}



\begin{frame}
\frametitle{7. Est ce que les IV sont importants? Quel est le niveau optimal pour monter un Pok\'emon?}

\begin{block}{}
\begin{footnotesize}

Concernant les IV, ils dépendent les statistiques de base du Pok\'emon.
\begin{itemize}
  \item Si un Pok\'emon a une statistique de base d'attaque importante, il n'y aura pas de grande différence entre ATK\_IV = 14 et ATK\_IV = 15. 
  
  Il est recommandé d'avoir une haute ATK\_IV pour un Pokémon avec un faible DPS.
  \item Concernant la résistance (DEF\_IV et HP\_IV), la statistique de défense (DEF) influence la vitesse à laquelle le Pokémon va mourrir. Généralement, il est plus intéressant d'avoir une plus haute DEF\_IV que HP\_IV. 
  
  Il est recommandé d'avoir une haute DEF\_IV et HP\_IV pour un Pokémon avec une faible vie effective. 
\end{itemize}

Il devient très cher de monter un Pok\'emon jusqu'au niveau 40; le niveau 35 semble excellent (le Pokémon est à 96.4\% de ses capacités), mais le niveau 30 est très correct (92.6\% de ses capacités). 

\begin{center}
Niveau 20 $\xrightarrow[\text{\stardust 31,000}]{\text{\candy~28}}$ Niveau 25 $\xrightarrow[\text{\stardust 44,000}]{\text{\candy~38}}$ Niveau 30 $\xrightarrow[\text{\stardust 62,000}]{\text{\candy~64}}$ Niveau 35 $\xrightarrow[\text{\stardust 88,000}]{\text{\candy~118}}$ Niveau 40
\end{center}

Il existe un niveau optimal pour chaque Pok\'emon (et il est lié à son opposant):
\begin{itemize}
  \item à partir d'un certain niveau, il ne sera plus résistant (chaque opposant peu infliger un certain nombre d'attaques rapides et chargées, au délà d'un certain niveau, ce nombre n'augmentera plus). On l'appelle le \textit{bulkpoint}.
  \item à partir d'un certain niveau (généralement indépendant du bulkpoint), l'attaque rapide de votre Pok\'emon fera exactement le même nombre de dégâts (c'est dû à la partie entière dans la formule des dégâts). On l'appelle le \textit{breakpoint}.
\end{itemize}

\textbf{Pour ceux qui montent leurs Pokémon au-delà du niveau 35}, si la statistique de base $\approx$ 125, la différence entre un IV 15 et un IV 14 est de \textbf{un niveau}. Si la statistique de base$\approx$ 265, la différence entre un IV 15 et un IV 14 is \textbf{un demi-niveau}.  

%\textbf{7. How are computed based statistics for a Pok\'emon?}

%Read \texttt{https://pokemongohub.net/post/meta/pokemon-go-cp-rework-stat-changes-formula-and-raiding-after-the-rework/}

\end{footnotesize}
\end{block}
\end{frame}


\end{document}
